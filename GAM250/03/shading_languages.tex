\part{Shading Languages}
\frame{\partpage}


\begin{frame}{High Level Shading Language (HLSL)}
\begin{itemize}
	\item Used for writing \textbf{shaders} for Direct3D and Unity3D
	\pause\item C-like syntax
	\pause\item But has data types that support mathematical operations
\end{itemize}
\end{frame}

\begin{frame}{Programming in HLSL}
	\begin{itemize}
		\item \lstinline{if} statements, \lstinline{for} loops, \lstinline{while} loops, \lstinline{do while} loops, \lstinline{switch} statements,
		\lstinline{break}, \lstinline{continue}, \lstinline{return} all work the same as C++
		\pause\item \lstinline{// Single-line comments} and \lstinline{/* Multi-line comments */} work the same too
		\pause\item Function definitions and declarations are similar to C\#, except that parameters must be declared as
		\lstinline{in}, \lstinline{out} or \lstinline{inout}
		\pause\item Recursion is \textbf{forbidden}
		\pause\item No \lstinline{class}
	\end{itemize}
\end{frame}

\begin{frame}{Data types in HLSL}
	\begin{itemize}
		\item \lstinline{bool}, \lstinline{int}, \lstinline{float}: just like in C++
		\pause\item \lstinline{float2}, \lstinline{float3}, \lstinline{float4}: \textbf{vectors} of \lstinline{float}s
		\pause\item \lstinline{float2x2}, \lstinline{float3x3}, \lstinline{float4x4}: \textbf{square matrices} of \lstinline{float}s
		\pause\item \textbf{Arrays} of constant size e.g.\ \lstinline{float myArray[10]}
	\end{itemize}
\end{frame}

\begin{frame}{Vectors}
	\begin{itemize}
		\item An \textbf{$n$-dimensional vector} is formed of $n$ numbers
		\pause\item E.g.\ 2-dimensional vectors:
		$$ (1, 2) \qquad (-2.7, 0) \qquad (3.4, -12.7) $$
		\pause\item E.g.\ 3-dimensional vectors:
		$$ (1, 2, 0) \qquad (-9, 6, 3.7) \qquad (2.1, 2.1, 2.1) $$
		\pause\item Used to represent \textbf{points} or \textbf{directions} in $n$ dimensions
		\pause\item Also used to represent e.g.\ colours in RGB(A) space
	\end{itemize}
\end{frame}

\begin{frame}[fragile]{Constructing vectors in GLSL}
	\begin{lstlisting}
	float3 a = float3(1.2, 3.4);
	float3 b = float3(1); // same as float3(1, 1, 1)
	float3 c = float3(a, 5.6); // same as float3(1.2, 3.4, 5.6)
	\end{lstlisting}
\end{frame}

\begin{frame}[fragile]{Vector maths}
	\pause Most operations work \textbf{component-wise}:
	\begin{lstlisting}
	float2 a = float2(1, 2);
	float2 b = float2(3, 4);
	float2 c = a + b; // c == float2(4, 6);
	float2 d = a * b; // d == float2(3, 8);
	\end{lstlisting}
	\pause Can also multiply a \textbf{vector} by a \textbf{scalar}:
	\begin{lstlisting}
	float2 e = 3.1 * a; // e == float2(3.1, 6.2)
	\end{lstlisting}
\end{frame}

\begin{frame}[fragile]{Accessing components}
	\pause Can access the components of a vector as \lstinline{.x}, \lstinline{.y}, \lstinline{.z}, \lstinline{.w}:
	\pause\begin{lstlisting}
	float4 a = float4(1, 2, 3, 4);
	float b = a.y; // b == 2
	float c = a.z; // c == 3
	a.x = 5;       // a == float4(5, 2, 3, 4)
	a.w = a.y;     // a == float4(5, 2, 3, 2)
	\end{lstlisting}
	\pause Can also use \lstinline{r g b a} (for colours) and \lstinline{t u v w} (for texture coordinates)
\end{frame}

\begin{frame}[fragile]{Swizzling}
	\pause Can access multiple components in one go:
	\pause\begin{lstlisting}
	float4 a = float4(1, 2, 3, 4);
	float2 b = a.xy;    // b == float2(1, 2)
	float3 c = a.zyz;   // c == float3(3, 2, 3)
	a.xw = float2(5,6); // a == float4(5, 2, 3, 6)
	a.xyzw = a.wzyx;  // a == float4(6, 3, 2, 5)
	\end{lstlisting}
	\begin{itemize}
		\pause \item\textbf{Can} use the same component twice in the \textbf{right-hand side} of an assignment 
		\pause \item\textbf{Cannot} use the same component twice in the \textbf{left-hand side} of an assignment 
		\pause \item Swizzling is generally \textbf{faster} than the equivalent code without swizzling
		\pause \item Can also use  \lstinline{r g b a} or \lstinline{t u v w}, but can't mix them
		(e.g.\ \lstinline{.gbr} is valid but \lstinline{.gzx} is not)
	\end{itemize}
\end{frame}

\begin{frame}{Texture Data Types}
	\begin{itemize}
	\pause\item Textures are stored in the \textbf{Sampler} data type
	\pause\item There are different samplers for different types of texture
	\begin{itemize}
		\pause\item 1D Texture - \textbf{sampler1D}
		\pause\item 2D Texture - \textbf{sampler2D}
		\pause\item 3D Texture - \textbf{sampler3D}
		\pause\item Cube Map - \textbf{samplerCube} 
	\end{itemize}
	\end{itemize}
\end{frame}

\begin{frame}{Unity Types}
	\begin{itemize}
		\item NB. When writing shaders you can used different precision data types rather than float (High precision)
		\begin{itemize}
			\pause\item Medium precision: \textbf{half} - directions, positions   
			\pause\item Low precision: \textbf{fixed} - colours
		\end{itemize}
		\pause\item On Desktop PCs these are always converted to high precision
		\pause\item These are important for optimisation for mobile
	\end{itemize}
\end{frame}

\begin{frame}{Surface Shader}
	\begin{center}
		Live Coding
	\end{center}
\end{frame}