\part{Subsurface Shaders}
\frame{\partpage}


\begin{frame}{Shaders in Unity}
	\begin{itemize}
		\item There are many approaches to writing shaders in Unity
		\begin{itemize}
			\pause\item Surface Shaders
			\pause\item Vertex and Fragment Shaders
			\pause\item Fixed Function Shaders
		\end{itemize}
		\pause\item The best method is to use Surface Shaders, this is the quickest way to get started
		\pause\item This interacts with the standard lights and shadows in Unity
		\pause\item Regardless of the shader type, your code will be wrapped in ShaderLab 
	\end{itemize}
\end{frame}

\begin{frame}{ShaderLab}
	\begin{itemize}
		\item ShaderLab is a simple scripting language for defining graphical effects
		\pause\item It contains the following
		\begin{itemize}
			\pause\item Properties - These are shown in the inspector of the material and is a way to expose shader variables
			\pause\item SubShaders - Is a list of pass or the surface shader code itself
		\end{itemize}
	\end{itemize}
\end{frame}

