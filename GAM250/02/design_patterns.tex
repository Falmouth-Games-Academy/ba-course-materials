\part{OO Design Basics}
\frame{\partpage}

\begin{frame}{Role of Design Patterns}
Object orientated systems tend to exhibit recurring structures that promote:

	\begin{itemize}
		\item Abstraction
		\item Flexibility
		\item Modularity
		\item Elegance
	\end{itemize}
\end{frame}

\begin{frame}{Role of Design Patterns}
	\begin{itemize}
		\item Therein lies valuable design knowledge.
		\item The challenge, of course, is to...		
		\begin{itemize}
			\item capture
			\item communicate
			\item and apply
		\end{itemize}
		\item ...this knowledge.
	\end{itemize}
\end{frame}

\begin{frame}{Role of Design Patterns}
A design pattern...

	\begin{itemize}
		\item Abstracts a recurring design structure
		\item Comprises class and/or object	
		\begin{itemize}
			\item dependencies
			\item structures
			\item interactions
			\item conventions
		\end{itemize}
		\item names and specifies the design structure explicitly
		\item and thereby distils design experience
	\end{itemize}
\end{frame}

\begin{frame}{Components of a Design Pattern}
A design pattern is comprised of:

	\begin{itemize}
		\item A name
		\item Common aliases --- \textit{also known as}...	
		\item Real-world examples
		\item Contexts
		\item Common problems solved
		\item Solution
		\item Structure
		\item Diagrams
		\item Consequences
	\end{itemize}
\end{frame}

\begin{frame}{Components of a Design Pattern}
	\begin{itemize}
		\item Design patterns are often tacit knowledge made explicit.
		\item You will develop tacit knowledge of patterns through regular design practice.
		\item You are expected to engage in constant research and reflection
		when designing software to learn all of these different patterns.
		\item They will help you communicate and design in the future.
		\item Additional research will be required as the number of patterns greatly
		exceeds those that can be covered in workshops.
	\end{itemize}
\end{frame}

\part{Design Patterns}
\frame{\partpage}

\begin{frame}{Types of Design Pattern}
Design patterns come in three main flavours:

	\begin{itemize}
		\item \textbf{creational}: concerned with the process of creating and managing the creation of objects.
		\item \textbf{structural}: dealing with the composition of objects.
		\item \textbf{behavioural}: characterizing the different means by which objects can interact with others.
	\end{itemize}
\end{frame}

\begin{frame}{Types of Design Pattern}
	\begin{columns}[onlytextwidth]
		\begin{column}{0.33\textwidth}
			\begin{itemize}
				\item \textbf{Creational}
				\item Singleton
				\item Typesafe Enum
				\item Factory
				\item Prototype
				\item Builder
			\end{itemize}
		\end{column}
		\begin{column}{0.33\textwidth}
			\begin{itemize}
				\item \textbf{Structural}
				\item Adapter
				\item Bridge
				\item Proxy
				\item Facade
				\item Decorator
			\end{itemize}
		\end{column}
		\begin{column}{0.33\textwidth}
			\begin{itemize}
				\item \textbf{Behavioural}
				\item Template
				\item State
				\item Observer
				\item Visitor				
				\item Strategy
			\end{itemize}
		\end{column}
	\end{columns}
\end{frame}

\begin{frame}{Design Patterns}
	We will now briefly examine these patterns. Throughout this section...
	
	\begin{itemize}
		\item \textbf{Please} make notes
		\item \textbf{Link} to on-line resources
		\item \textbf{Ask} questions
		\item \textbf{Think} about how the patterns may apply to your own projects
		\item \textbf{Conduct} further research
	\end{itemize}
\end{frame}

\begin{frame}{Singleton}
	\begin{itemize}
		\item Guarantees that there is only one instance of a class and can be accessed globally
		\item Usually 'lazily' initialised via a static function that satisfy the statement above
		\item Used for manager classes which track some sort of Global State
		\item \textbf{Warning!} Some consider Singletons to be an anti-pattern
		\item Singleton: an anti-pattern? - \url{https://stackoverflow.com/questions/12755539/why-is-singleton-considered-an-anti-pattern}
	\end{itemize}
\end{frame}


\begin{frame}{Abstract Factory}
	\begin{itemize}
		\item Centralises the creation of similar objects
		\item Decouples the creation of the object from actual object
		\item This pattern requires several class
		\begin{itemize}
			\item Abstract Product - Base class for all things created by the Factory
			\item Abstract Factory - Base class for all factories, creates Abstract Products
			\item Many Concrete Products - Implement Abstract Product
			\item Many Concrete Factories - Implements Abstract Factory and creates Concrete Products 
		\end{itemize}
		\item The caller then creates instances of Product through the concrete factory
		\item Used for spawning objects or the creation of other similar objects
	\end{itemize}
\end{frame}

\begin{frame}{Observer}
	\begin{itemize}
		\item When one object is updated, all observers of this object are notified
		\item A list of observers are mainted by the subject
		\item When the state of the subject changes then the list of the observers is processed
		\item Each observer is then notified of the change
		\item Each observer should register/unregistered itself with a subject 
		\item Very useful for UI, Input or Network systems in games
		\item Some of this function is already built into C\#(delegates \& Events) and Unity(Unity Events)
	\end{itemize}
\end{frame}

\begin{frame}{State}
	\begin{itemize}
		\item Do you have large amount of if..else or switch statements in your code?
		\item Have you ever had to change such a system?
		\item Then the State pattern is here to help
		\item You define a Base State class which all other States implement
		\item This Base State will have a method for updating the state, for entering and exiting
		\item Each Concrete State will then implement these methods and handle its own logic 
		\item Transitions can be handled by a Manager class
		\item This is generally used to deal with Game State or AI (see Finite State Machines)
	\end{itemize}
\end{frame}

\begin{frame}{Unity Implementations}
	\begin{itemize}
		\item Singleton - \url{https://unity3d.com/learn/tutorials/projects/2d-roguelike-tutorial/writing-game-manager}
		\item Factory - \url{http://brightreasongames.com/object-construction-factory-method/}
		\item State - \url{http://www.habrador.com/tutorials/programming-patterns/6-state-pattern/}
		\item Observer -
		\url{http://www.habrador.com/tutorials/programming-patterns/3-observer-pattern/}
		
	\end{itemize}
\end{frame}