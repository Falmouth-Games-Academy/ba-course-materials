\part{Attributes}
\frame{\partpage}

\begin{frame}{Attributes in C\#}
	\begin{itemize}
		\item These are classes that allow you to add metadata to other classes
		\item We can then extract these attributes and use them in our own code
		\item There are a few of these defined in the C\# language including
		\begin{itemize}
			\item Obsolete - Mark a method as obsolete
			\item Conditional - Mark a method as optionable based on a flag
			\item AttributeUsage - Defines how a custom attribute can be used
		\end{itemize}
		\item We can also create our own custom attributes by creating a class that inherits from \textbf{Attribute}
	\end{itemize}
\end{frame}

\begin{frame}[fragile]{Using Attributes}
\begin{itemize}
	\item To use an attribute you \textbf{decorate} a method, class or variable with a set of square brackets with the name of the attribute
\end{itemize}
\begin{lstlisting}
	[Obsolete]
	void OldUpdateMethod(int x,int y)
\end{lstlisting}
\begin{itemize}
	\item Please note attributes can decorate classes, methods or variables
\end{itemize}
\end{frame}

\begin{frame}{Unity Attributes}
\begin{itemize}
	\item Unity has a number of inbuilt Attributes that can be used to mark your code (or inherit from to make new Attributes)
	\begin{itemize}
		\pause\item ContexMenu - Markup a function - Adds a command to the context menu which can be selected in the inspector.
		\pause\item SerializeField - Markup a Variable - Forces Unity to serialize a variable which will then be displayed in the inspector
		\pause\item CustomEditor - Markup a class which inherits from Editor - This specifies that this editor class will be an editor for a certain script.
		\pause\item DrawGizmo - Markup a function - This method should have same signature as DrawGizmo, will draw a gizmo in the editor.  
	\end{itemize} 
	\pause\url{https://docs.unity3d.com/400/Documentation/ScriptReference/20_class_hierarchy.Attributes.html}
\end{itemize}
\end{frame}

\begin{frame}{Property Drawer}
\begin{itemize}
	\item Property Drawers allow you to customize how a script displays its variables in the inspector
	\item You have to create a class that inherits from \textbf{PropertyDrawer}, then override the \textbf{OnGUI} function
	\item We then have complete control of how the property is drawn using Immediate Mode GUI (drawing via code)
\end{itemize}
\end{frame}