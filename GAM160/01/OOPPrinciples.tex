\part{Object Orientated Principles}
\frame{\partpage}

\begin{frame}{Classes and Objects}
	\begin{itemize}
		\pause \item Most programming languages have a pre-defined set of data types (int, float, bool etc)
		\pause \item We can add our own data types by declaring and defining Classes
		\pause \item Classes are a collection of data and functions which operate on the data
		\pause \item We can then use these classes like any built-in data type
	\end{itemize}
\end{frame}

\begin{frame}[fragile]{Class Examples - C++}
	\begin{lstlisting}[language=C++]
	class Player
	{
	public:
		Player()
		{
			Health=100;
		};
		
		void TakeDamage(int health)
		{
			Health-=health;
		};
		
		void HealDamage(int health)
		{
			Health+=health;
		};
		
		~Player(){};
	private:
		int Health;
		
	};
	\end{lstlisting}
\end{frame}

\begin{frame}[fragile]{Class Examples - C\# Unity}
	\begin{lstlisting}[language=C++]
	class Player : MonoBehaviour
	{
		private int Health=100;
		
		void Start()
		{
			Health=100;
		}
	
		public void TakeDamage(int health)
		{
			Health-=health;
		}
	
		public void HealDamage(int health)
		{
			Health+=health;
		}
	}
	\end{lstlisting}
\end{frame}

\begin{frame}{Naming Tips}
	\begin{itemize}
		\pause \item Classnames should be proper nouns (e.g. Player, Enemy, Orc, Goblin, etc)
		\pause \item Function names should be verbs (e.g Attack, TakeDamage, SetName, etc)
		\pause \item All names should be descriptive
		\pause \item Comments! You should add comments before each function!
	\end{itemize}
\end{frame}

