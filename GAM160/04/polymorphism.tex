\part{Polymorphism}
\frame{\partpage}

\begin{frame}{Introduction to Polymorphism}
	\begin{itemize}
		\pause \item Polymorphism is another key feature of OOP languages
		\pause \item The basic idea is that instances of a derived class can be treated as objects of the basic class
		\pause \item They can be used as parameters for functions and in collections
		\pause \item We then call the functions on these objects and our code will called the 'correct' version of the function
		\pause \item This is best illustrated by an example 
	\end{itemize}
\end{frame}

\begin{frame}[fragile]{Polymorphism example C\#}
		\begin{lstlisting}[language=C++,basicstyle=\tiny,]
		class Enemy{/*This has been defined in previous slides*/}
		class Boss : Enemy{/*Again see previou slides*/}
		
		//This function will be in monobehavior
		void DoAttacks(Enemy enemy)
		{
			enemy.Attack();
		}
		
		//We probably have grabbed these from other game objects
		Enemy goblin=new Enemy();
		Eneny orc=new Enemy();
		Boss ogre=new Boss();
		
		//Call DoAttack on each one of these
		DoAttack(goblin);
		DoAttack(orc);
		DoAttack(ogre);
		\end{lstlisting}
\end{frame}


\begin{frame}{Abstract Classes \& Interfaces}
\begin{itemize}
	\pause \item An \textbf{Abstract Class} is a class which cannot be initialised but is intended to be used as a base class
	\pause \item It will have at lease one function marked as pure virtual (see example)
	\pause \item If you then inherit from an abstract class, you have to provide an implementation of all pure virtual functions
\end{itemize}
\end{frame}

\begin{frame}{Abstract Classes \& Interfaces}
\begin{itemize}
\pause \item An \textbf{Interface} is very similar to an abstract class, the only difference is that every function in an Interface is marked as pure virtual
\pause \item If you then inherit from an interface, you have to provide an implementation of all pure virtual functions
\pause \item In C\# you can't inherit from multiple Classes or Abstract Classes, however you can inherit from multiple Interfaces.
\end{itemize}
\end{frame}


\begin{frame}[fragile]{Abstract Class Example C\#}
	\begin{lstlisting}[language=C++,basicstyle=\tiny,]
	public abstract class BaseEnemy
	{
		public abstract public void Attack();
		
		public void Jump()
		{
			//Do jump code
		}
	}
	
	public class Orc : BaseEnemy
	{
		//we have to implement attack but no need to implement Jump
		public void Attack()
		{
			//do attack
		}
	}
	\end{lstlisting}
\end{frame}

\begin{frame}[fragile]{Interface Example C\#}
	\begin{lstlisting}[language=C++,basicstyle=\tiny,]
	interface Jump
	{
		void DoJump();
	}
	
	interace Attack
	{
		void DoAttack();
	}
	
	public class Orc : Jump, Attack
	{
		//we have to implement Attack and Jump Interface
		public void DoAttack()
		{
			//do attack
		}
		
		public void DoJump()
		{
			//do Jump
		}
	}
	\end{lstlisting}
\end{frame}


\begin{frame}{Interface Discussion}
\begin{itemize}
	\pause \item You can think of an Interface as a contract
	\pause \item The derived class must implement the Interface's function
	\pause \item We can leverage Polymorphism to work with interfaces
	\pause \item This means that I can consume derived classes in a function that takes in references to the Interface
\end{itemize}
\end{frame}

\begin{frame}{Interface Discussion}
\begin{itemize}
\pause \item Lastly, Interfaces a great tool for working with others. We as a group could create the interface together
\pause \item Then another programmer can write Classes which implement the Interface
\pause \item While another writes code which consumes instances of the Interface 
\pause \item \url{https://stackoverflow.com/questions/4456424/what-do-programmers-mean-when-they-say-code-against-an-interface-not-an-objec} 
\end{itemize}
\end{frame}